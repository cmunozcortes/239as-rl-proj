\documentclass[landscape,a0paper,fontscale=0.292]{baposter}

\usepackage[vlined]{algorithm2e}
\usepackage{times}
\usepackage{calc}
\usepackage{url}
\usepackage{graphicx}
\usepackage{amsmath}
\usepackage{amssymb}
\usepackage{relsize}
\usepackage{multirow}
\usepackage{booktabs}

\usepackage{graphicx}
\usepackage{multicol}
\usepackage[T1]{fontenc}
\usepackage{ae}

\graphicspath{{images/}}

 %%%%%%%%%%%%%%%%%%%%%%%%%%%%%%%%%%%%%%%%%%%%%%%%%%%%%%%%%%%%%%%%%%%%%%%%%%%%%%%%
 %%%% Some math symbols used in the text
 %%%%%%%%%%%%%%%%%%%%%%%%%%%%%%%%%%%%%%%%%%%%%%%%%%%%%%%%%%%%%%%%%%%%%%%%%%%%%%%%
 % Format 
 \newcommand{\RotUP}[1]{\begin{sideways}#1\end{sideways}}


 %%%%%%%%%%%%%%%%%%%%%%%%%%%%%%%%%%%%%%%%%%%%%%%%%%%%%%%%%%%%%%%%%%%%%%%%%%%%%%%%
 % Multicol Settings
 %%%%%%%%%%%%%%%%%%%%%%%%%%%%%%%%%%%%%%%%%%%%%%%%%%%%%%%%%%%%%%%%%%%%%%%%%%%%%%%%
 \setlength{\columnsep}{0.7em}
 \setlength{\columnseprule}{0mm}


 %%%%%%%%%%%%%%%%%%%%%%%%%%%%%%%%%%%%%%%%%%%%%%%%%%%%%%%%%%%%%%%%%%%%%%%%%%%%%%%%
 % Save space in lists. Use this after the opening of the list
 %%%%%%%%%%%%%%%%%%%%%%%%%%%%%%%%%%%%%%%%%%%%%%%%%%%%%%%%%%%%%%%%%%%%%%%%%%%%%%%%
 \newcommand{\compresslist}{%
 \setlength{\itemsep}{1pt}%
 \setlength{\parskip}{0pt}%
 \setlength{\parsep}{0pt}%
 }


 %%%%%%%%%%%%%%%%%%%%%%%%%%%%%%%%%%%%%%%%%%%%%%%%%%%%%%%%%%%%%%%%%%%%%%%%%%%%%%
 % Formating
 \newcommand{\Matrix}[1]{\begin{bmatrix} #1 \end{bmatrix}}
 \newcommand{\Vector}[1]{\begin{pmatrix} #1 \end{pmatrix}}

 \newcommand*{\norm}[1]{\mathopen\| #1 \mathclose\|}% use instead of $\|x\|$
 \newcommand*{\abs}[1]{\mathopen| #1 \mathclose|}% use instead of $\|x\|$
 \newcommand*{\normLR}[1]{\left\| #1 \right\|}% use instead of $\|x\|$

 \newcommand*{\SET}[1]  {\ensuremath{\mathcal{#1}}}
 \newcommand*{\FUN}[1]  {\ensuremath{\mathcal{#1}}}
 \newcommand*{\MAT}[1]  {\ensuremath{\boldsymbol{#1}}}
 \newcommand*{\VEC}[1]  {\ensuremath{\boldsymbol{#1}}}
 \newcommand*{\CONST}[1]{\ensuremath{\mathit{#1}}}

 \DeclareMathOperator*{\argmax}{arg\,max}
 \DeclareMathOperator*{\diag}{diag}
 \DeclareMathOperator*{\argmin}{arg\,min}
 \DeclareMathOperator*{\vectorize}{vec}
 \DeclareMathOperator*{\reshape}{reshape}

 %-----------------------------------------------------------------------------
 % Differentiation
 \newcommand*{\Nabla}[1]{\nabla_{\!#1}}

 \renewcommand*{\d}{\mathrm{d}}
 \newcommand*{\dd}{\partial}

 \newcommand*{\At}[2]{\ensuremath{\left.#1\right|_{#2}}}
 \newcommand*{\AtZero}[1]{\At{#1}{\pp=\VEC 0}}

 \newcommand*{\diffp}[2]{\ensuremath{\frac{\dd #1}{\dd #2}}}
 \newcommand*{\diffpp}[3]{\ensuremath{\frac{\dd^2 #1}{\dd #2 \dd #3}}}
 \newcommand*{\diffppp}[4]{\ensuremath{\frac{\dd^3 #1}{\dd #2 \dd #3 \dd #4}}}
 \newcommand*{\difff}[2]{\ensuremath{\frac{\d #1}{\d #2}}}
 \newcommand*{\diffff}[3]{\ensuremath{\frac{\d^2 #1}{\d #2 \d #3}}}
 \newcommand*{\difffp}[3]{\ensuremath{\frac{\dd\d #1}{\d #2 \dd #3}}}
 \newcommand*{\difffpp}[4]{\ensuremath{\frac{\dd^2\d #1}{\d #2 \dd #3 \dd #4}}}

 \newcommand*{\diffpAtZero}[2]{\ensuremath{\AtZero{\diffp{#1}{#2}}}}
 \newcommand*{\diffppAtZero}[3]{\ensuremath{\AtZero{\diffpp{#1}{#2}{#3}}}}
 \newcommand*{\difffAt}[3]{\ensuremath{\At{\difff{#1}{#2}}{#3}}}
 \newcommand*{\difffAtZero}[2]{\ensuremath{\AtZero{\difff{#1}{#2}}}}
 \newcommand*{\difffpAtZero}[3]{\ensuremath{\AtZero{\difffp{#1}{#2}{#3}}}}
 \newcommand*{\difffppAtZero}[4]{\ensuremath{\AtZero{\difffpp{#1}{#2}{#3}{#4}}}}

 %-----------------------------------------------------------------------------
 % Defined
 % How should the defined operator look like (:= or ^= ==)
 % (I want back my :=, it is so much better than ^= because (1) it has a
 % direction and (2) everyone here uses it.)
 %
 % Use :=
 %\newcommand*{\defined}{\ensuremath{\mathrel{\mathop{:}}=}}
 %\newcommand*{\definedRight}{\ensuremath{=\mathrel{\mathop{:}}}}
 % Use ^=
 \newcommand*{\defined}{\ensuremath{\triangleq}}
 \newcommand*{\definedRight}{\ensuremath{\triangleq}}
 % Use = with three bars
 %\newcommand*{\defined}{\ensuremath{?}}
 %\newcommand*{\definedRight}{\ensuremath{?}}

 %%%%%%%%%%%%%%%%%%%%%%%%%%%%%%%%%%%%%%%%%%%%%%%%%%%%%%%%%%%%%%%%%%%%%%%%%%%%%%
 % Symbols used in the paper

 %-----------------------------------------------------------------------------
 % The Methods
 \newcommand*{\ICIA}{\emph{ICIA}}
 \newcommand*{\CoDe}{\emph{CoDe}}
 \newcommand*{\LinCoDe}{\emph{LinCoDe}}
 \newcommand*{\CoNe}{\emph{CoNe}}
 \newcommand*{\CoLiNe}{\emph{CoLiNe}}
 \newcommand*{\LinCoLiNe}{\emph{LinCoLiNe}}

 % inter eye distance
 \newcommand*{\ied}{IED}

 %-----------------------------------------------------------------------------
 % Koerper
 %%\newcommand*{\RR}{\mathbb{R}}
 %\newcommand*{\RR}{{I\hspace{-3.5pt}R}}
 %\newcommand*{\RR}{{\mathrm{I\hspace{-2.7pt}R}}}

 \font\dsfnt=dsrom12

 \DeclareSymbolFont{nark}{U}{dsrom}{m}{n}
 \DeclareMathSymbol{\NN}{\dsfnt}{nark}{`N}
 \DeclareMathSymbol{\RR}{\dsfnt}{nark}{`R}
 \DeclareMathSymbol{\ZZ}{\dsfnt}{nark}{`Z}

 %-----------------------------------------------------------------------------
 % Domains
 \newcommand*{\D}{\mathcal{D}}
 \newcommand*{\I}{\mathcal{I}}

 %-----------------------------------------------------------------------------
 % Texture coordinates
 \newcommand*{\rr}{\VEC{r}}

 %-----------------------------------------------------------------------------
 % Parameters
 \newcommand*{\pt}{\VEC{\tau}}
 \newcommand*{\pr}{\VEC{\rho}}
 \newcommand*{\pp}{\VEC{p}}
 \newcommand*{\qq}{\VEC{q}}
 \newcommand*{\xx}{\VEC{x}}
 \newcommand*{\deltaq}{\Delta \qq}
 \newcommand*{\deltap}{\Delta \pp}
 \newcommand*{\zz}{\VEC{z}}
 \newcommand*{\pa}{\VEC{\alpha}}
 \newcommand*{\qa}{\VEC{\alpha}}
 \newcommand*{\pb}{\VEC{\beta}}

 %-----------------------------------------------------------------------------
 % Optimal appearance parameters
 \newcommand*{\pbh}[1]{\ensuremath{\hat{\pb}({#1})}}

 %-----------------------------------------------------------------------------
 % Warp basis
 \newcommand*{\M}[1]{\ensuremath{M({#1})}}
 \newcommand*{\LL}[1]{\ensuremath{L({#1})}}

 %-----------------------------------------------------------------------------
 % Matrices of the texture model
 \newcommand*{\AM}[1]{\ensuremath{\Lambda(#1)}}               % Lambda(beta) 
 \newcommand*{\AMr}[2]{\ensuremath{\Lambda(#1; #2)}}        % Lambda(r, beta)

 \newcommand*{\As}{A}         % Continuous Basis symbol
 \newcommand*{\afs}{a}        % Continuous mean symbol
 \newcommand*{\A}[1]{\As(#1)}         % Continuous Basis
 \newcommand*{\af}[1]{\afs(#1)}        % Continuous mean


 %-----------------------------------------------------------------------------
 % Matrices of the shape model
 \newcommand*{\MU}{\VEC{\mu}}
 \newcommand*{\MM}{\MAT{M}}

 %-----------------------------------------------------------------------------
 %% The project out matrix and operator
 \newcommand*{\INT}{\MAT{P}}
 \newcommand*{\INTf}{P}

 %-----------------------------------------------------------------------------
 % The identity matrix
 \newcommand*{\EYEtwo}{\Matrix{1 & 0\\0&1}}
 \newcommand*{\EYE}{\MAT E}
 \newcommand*{\EYEf}{E}

 % Wether to use subscripts or brackets for some function arguments
 % can be decided by commenting out the corresponding functions underneath
 %-----------------------------------------------------------------------------
 % Mapping
 \newcommand*{\Cs}[1]{\ensuremath{C^{#1}}} % C symbol
 \newcommand*{\C}[2]{\ensuremath{C^{#1}(#2)}} % Use C with brackets

 %-----------------------------------------------------------------------------
 % Objective function
 \newcommand*{\Fs}{\ensuremath{F}}              % F symbol
 \newcommand*{\F}[1]{\ensuremath{\Fs(#1)}}       % Use F with brackets    F(q)

 %-----------------------------------------------------------------------------
 % Approximated objective functions
 \newcommand*{\FFs}{\tilde{F}}                     % ~F symbol
 \newcommand*{\FF}[1]{\ensuremath{\FFs(#1)}}       % Use ~F with brackets    F(q)

 %-----------------------------------------------------------------------------
 % residual function
 \newcommand*{\es}{\ensuremath{f}}              % R symbol

 \newcommand*{\e}[1]{\ensuremath{\es(#1)}}         % R(q)
 \newcommand*{\er}[2]{\ensuremath{\es(#1; #2)}}    % R(r; q)

 %-----------------------------------------------------------------------------
 % Approximated residual functions
 \newcommand*{\ees}{\tilde{f}}                       % ~R symbol
 \newcommand*{\ee}[1]{\ensuremath{\ees(#1)}}       % ~R(q)
 \newcommand*{\eer}[2]{\ensuremath{\ees(#2; #1)}}  % ~R(r; q)

 %-----------------------------------------------------------------------------
 % Warps
 \newcommand*{\Vs}{\ensuremath{V}}
 \newcommand*{\VLins}{\ensuremath{\Vs^{\text{Ortho}}}}
 \newcommand{\VModels}{\ensuremath{\Vs^{\text{Model}}}}
 \newcommand*{\Ws}{\ensuremath{W}}

 \newcommand{\V}[1]{\ensuremath{\Vs(#1)}}
 \newcommand{\VModel}[1]{\ensuremath{\VModels(#1)}}
 \newcommand{\Vr}[2]{\ensuremath{\Vs(#1; #2)}}
 \newcommand{\VInvr}[2]{\ensuremath{\Vs^{-1}(#1; #2)}}
 \newcommand{\VrLin}[2]{\ensuremath{\VLins(#1; #2)}}
 \newcommand{\W}[1]{\ensuremath{\Ws(#1)}}
 \newcommand{\Winv}[1]{\ensuremath{\Ws^{-1}(#1)}}
 \newcommand{\Wr}[2]{\ensuremath{\Ws(#1; #2)}}


%%%%%%%%%%%%%%%%%%%%%%%%%%%%%%%%%%%%%%%%%%%%%%%%%%%%%%%%%%%%%%%%%%%%%%%%%%%%%
%% Begin of Document
%%%%%%%%%%%%%%%%%%%%%%%%%%%%%%%%%%%%%%%%%%%%%%%%%%%%%%%%%%%%%%%%%%%%%%%%%%%%%
\begin{document}
%%%%%%%%%%%%%%%%%%%%%%%%%%%%%%%%%%%%%%%%%%%%%%%%%%%%%%%%%%%%%%%%%%%%%%%%%%%%%
%% Here starts the poster
%%---------------------------------------------------------------------------
%% Format it to your taste with the options
%%%%%%%%%%%%%%%%%%%%%%%%%%%%%%%%%%%%%%%%%%%%%%%%%%%%%%%%%%%%%%%%%%%%%%%%%%%%%
\begin{poster}{
 % Show grid to help with alignment
 grid=false,
 % Column spacing
 colspacing=0.7em,
 % Color style
 headerColorOne=cyan!20!white!90!black,
 borderColor=cyan!30!white!90!black,
 % Format of textbox
 textborder=faded,
 % Format of text header
 headerborder=open,
 headershape=roundedright,
 headershade=plain,
 background=none,
 bgColorOne=cyan!10!white,
 headerheight=0.12\textheight}
 % Eye Catcher
 {
      \includegraphics[height=0.1\textheight]{ucla_campus_seal}
 }
 % Title
 {\sc\Huge A Brief Survey of Deep Q-Learning Methods}
 % Authors
 {Ryan Chau, My-Quan Hong, Nathan Kang, and Chris Munoz\\[0.5em]
 {\texttt{chau\_ryan@yahoo.com, myquan@yahoo.com, nkang@seis.ucla.edu,
 cmunozcortes@ucla.edu}}}
 % University logo
 {
  \begin{tabular}{r}
    \includegraphics[height=0.05\textheight]{logo_UCLA_blue}
  \end{tabular}
 }

%%%%%%%%%%%%%%%%%%%%%%%%%%%%%%%%%%%%%%%%%%%%%%%%%%%%%%%%%%%%%%%%%%%%%%%%%%%%%%
%%% Now define the boxes that make up the poster
%%%---------------------------------------------------------------------------
%%% Each box has a name and can be placed absolutely or relatively.
%%% The only inconvenience is that you can only specify a relative position 
%%% towards an already declared box. So if you have a box attached to the 
%%% bottom, one to the top and a third one which should be inbetween, you 
%%% have to specify the top and bottom boxes before you specify the middle 
%%% box.
%%%%%%%%%%%%%%%%%%%%%%%%%%%%%%%%%%%%%%%%%%%%%%%%%%%%%%%%%%%%%%%%%%%%%%%%%%%%%%

%%%%%%%%%%%%%%%%%%%%%%%%%%%%%%%%%%%%%%%%%%%%%%%%%%%%%%%%%%%%%%%%%%%%%%%%%%%%%%
  \headerbox{Contribution: Fast \emph{and} reliable Face Alignment}{name=contribution,column=0,row=0,span=2}{
%%%%%%%%%%%%%%%%%%%%%%%%%%%%%%%%%%%%%%%%%%%%%%%%%%%%%%%%%%%%%%%%%%%%%%%%%%%%%%
  Inverse compositional image alignment (ICIA) is fast, but not reliable. We
  explain ICIA from a different perspective which leads naturally to two new
  algorithms with a better capture range and comparable speed.
  }
%%%%%%%%%%%%%%%%%%%%%%%%%%%%%%%%%%%%%%%%%%%%%%%%%%%%%%%%%%%%%%%%%%%%%%%%%%%%%%
  \headerbox{There is no \emph{inverse} in \emph{ICIA}}{name=abstract,column=0,below=contribution}{
%%%%%%%%%%%%%%%%%%%%%%%%%%%%%%%%%%%%%%%%%%%%%%%%%%%%%%%%%%%%%%%%%%%%%%%%%%%%%%
    Additionally we replace the incremental warp $V$ with an orthonormalized
    warp and regularize in the composition step. The result is a vast
    improvement in robustness without sacrificing speed.
}



 %%%%%%%%%%%%%%%%%%%%%%%%%%%%%%%%%%%%%%%%%%%%%%%%%%%%%%%%%%%%%%%%%%%%%%%%%%%%%%
   \headerbox{Our methods are at the performance/speed sweet point}{name=speed,column=2,row=0,span=2}{
 %%%%%%%%%%%%%%%%%%%%%%%%%%%%%%%%%%%%%%%%%%%%%%%%%%%%%%%%%%%%%%%%%%%%%%%%%%%%%%
   \begin{multicols}{2}
   \textbf{Fitting a multiperson AAM. }
   The best speed--performance tradeoffs come from the two new algorithms
   \CoDe{} and \LinCoDe{}. Note that \ICIA{} is practically useless on
   this difficult multi-person dataset with a success rate near zero (left). It
   can be improved (right) by using the orthonormal incremental warp and
   regularisation. The \CoDe{} algorithm with regularisation (right) is as
   accurate as the slow, approximation-free, compositional Gauss-Newton \CoNe{}
   method but is seven times more efficient.

   The experiments were performed with leave one identity out on a mixture of two databases (XM2VTS and IMM).
   \end{multicols}
   }
 %%%%%%%%%%%%%%%%%%%%%%%%%%%%%%%%%%%%%%%%%%%%%%%%%%%%%%%%%%%%%%%%%%%%%%%%%%%%%%
   \headerbox{References}{name=references,column=0,above=bottom}{
 %%%%%%%%%%%%%%%%%%%%%%%%%%%%%%%%%%%%%%%%%%%%%%%%%%%%%%%%%%%%%%%%%%%%%%%%%%%%%%
     \smaller
     
     \bibliographystyle{ieee}
     \renewcommand{\section}[2]{\vskip 0.05em}
       \begin{thebibliography}{1}\itemsep=-0.01em
       \setlength{\baselineskip}{0.4em}

       \bibitem{amberg07:nicp}
       B.~Amberg, A.~Blake, T.~Vetter
       \newblock On Compositional Image Alignment with an Application to Activce Appearance Models
       \newblock In {\em CVPR'09}, 2009.

       \end{thebibliography}
   }

 %%%%%%%%%%%%%%%%%%%%%%%%%%%%%%%%%%%%%%%%%%%%%%%%%%%%%%%%%%%%%%%%%%%%%%%%%%%%%%
   \headerbox{Training + Testing Data}{name=data,column=0,above=references,below=abstract}{
 %%%%%%%%%%%%%%%%%%%%%%%%%%%%%%%%%%%%%%%%%%%%%%%%%%%%%%%%%%%%%%%%%%%%%%%%%%%%%%
   The model was trained from 456 images from the IMM and XM2VTS datasets using
   120 landmarks. Get the landmarks, model, and source code at:\\
   \mbox{\url{www.cs.unibas.ch/personen/amberg_brian/aam/}}
   }

 %%%%%%%%%%%%%%%%%%%%%%%%%%%%%%%%%%%%%%%%%%%%%%%%%%%%%%%%%%%%%%%%%%%%%%%%%%%%%%

   \headerbox{Tracking 5000 frames with a general model}{name=tracking,column=2,span=2,below=speed,above=bottom}{
 %%%%%%%%%%%%%%%%%%%%%%%%%%%%%%%%%%%%%%%%%%%%%%%%%%%%%%%%%%%%%%%%%%%%%%%%%%%%%%
   \begin{multicols}{2}
   {\textbf{Our algorithm makes fast and robust tracking possible.}}
   The same training dataset was used for both tracking experiments. The
   training data was aquired with different camera and light settings from
   different subjects.
   \end{multicols}
   }
 %%%%%%%%%%%%%%%%%%%%%%%%%%%%%%%%%%%%%%%%%%%%%%%%%%%%%%%%%%%%%%%%%%%%%%%%%%%%%%
   \headerbox{Low Res Tracking}{name=lowrestracking,column=1,span=1,below=speed,above=bottom}{
 %%%%%%%%%%%%%%%%%%%%%%%%%%%%%%%%%%%%%%%%%%%%%%%%%%%%%%%%%%%%%%%%%%%%%%%%%%%%%%
  \textbf{Tracking a low resolution video with large head motions
  succeeds with, where fails.}\\ All methods used the orthonormal
   }

 %%%%%%%%%%%%%%%%%%%%%%%%%%%%%%%%%%%%%%%%%%%%%%%%%%%%%%%%%%%%%%%%%%%%%%%%%%%%%%%
  \headerbox{Compositional Alignment}{name=algorithm,column=1,above=lowrestracking,below=contribution}{
 %%%%%%%%%%%%%%%%%%%%%%%%%%%%%%%%%%%%%%%%%%%%%%%%%%%%%%%%%%%%%%%%%%%%%%%%%%%%%%%
  \begin{algorithm}[H]
    \dontprintsemicolon
    \linesnumbered
    \For{Blur and regularisation values}{
    }
  \end{algorithm}
   }
\end{poster}%
%
\end{document}
